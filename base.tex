\documentclass{article}
\usepackage{amsmath}
\usepackage{algorithm}
\usepackage{algorithmic}

\begin{document}

\section*{Segmentación de Video basada en Diferencia de Histograma}

\subsection*{Algoritmo de Detección de Escenas basada en Diferencia de Histograma}

Este algoritmo detecta cambios de escena en un video al calcular la diferencia entre los histogramas de cuadros consecutivos. Un cambio significativo en la diferencia indica un cambio de escena.

\begin{enumerate}
    \item \textbf{Calcular Histogramas de Cuadros:} Sean \( I_t \) la imagen (cuadro) en el tiempo \( t \). Calculamos el histograma de la imagen, \( H_t \), que es un vector donde cada componente \( H_t(i) \) representa la frecuencia de píxeles en el bin \( i \).
    
    \[
    H_t = \text{histograma}(I_t)
    \]
    
    \item \textbf{Calcular la Diferencia entre Histogramas:} Calculamos la diferencia \( D_t \) entre los histogramas de cuadros consecutivos:
    
    \[
    D_t = \sum_{i=1}^{N} \left| H_t(i) - H_{t-1}(i) \right|
    \]
    
    donde \( N \) es el número de bins en el histograma.
    
    \item \textbf{Detectar Cambios de Escena:} Definimos un umbral \( T \) y detectamos un cambio de escena si \( D_t \) supera \( T \):
    
    \[
    \text{si} \ D_t > T \ \text{entonces hay un cambio de escena en} \ t
    \]
\end{enumerate}

\subsection*{Algoritmo}
\begin{algorithm}
\caption{Detectar Escenas}
\begin{algorithmic}[1]
\STATE \textbf{function} \textsc{detect\_scenes}(video\_path, threshold)
\STATE \quad abrir archivo de video en video\_path
\STATE \quad prev\_hist = histograma(primer cuadro)
\STATE \quad scene\_changes = []
\STATE \quad \textbf{for} cada cuadro en el video
\STATE \quad \quad curr\_hist = histograma(cuadro actual)
\STATE \quad \quad diff = \sum_{i=1}^{N} |curr\_hist[i] - prev\_hist[i]|
\STATE \quad \quad \textbf{if} diff > threshold \textbf{then}
\STATE \quad \quad \quad scene\_changes.append(tiempo actual)
\STATE \quad \quad prev\_hist = curr\_hist
\STATE \quad \textbf{return} scene\_changes
\end{algorithmic}
\end{algorithm}

\section*{Reconstrucción de Video basada en Ordenación y Concatenación}

\subsection*{Algoritmo de Ordenación y Concatenación}

Este algoritmo reensambla un video a partir de segmentos, ordenándolos según su posición original y concatenándolos para formar el video completo.

\begin{enumerate}
    \item \textbf{Ordenar Segmentos:} Sea \( S = \{S_1, S_2, \ldots, S_n\} \) el conjunto de segmentos de video, donde cada segmento \( S_i \) tiene una posición original \( p_i \) en el video original.
    
    Ordenamos los segmentos según \( p_i \):
    
    \[
    S_{\text{ordenado}} = \text{sort}(S, \text{by} \ p_i)
    \]
    
    \item \textbf{Concatenar Segmentos:} Concatenamos los segmentos ordenados para formar el video reconstruido \( V \):
    
    \[
    V = S_{\text{ordenado}}[1] + S_{\text{ordenado}}[2] + \ldots + S_{\text{ordenado}}[n]
    \]
    
    donde \( + \) denota la operación de concatenación de video.
\end{enumerate}

\subsection*{Algoritmo}
\begin{algorithm}
\caption{Reconstruir Video}
\begin{algorithmic}[1]
\STATE \textbf{function} \textsc{reassemble\_video}(segments)
\STATE \quad ordenar segments por posición original
\STATE \quad video = concatenar segments ordenados
\STATE \quad \textbf{return} video
\end{algorithmic}
\end{algorithm}

\section*{Matemáticas del Algoritmo de Interpolación de Transiciones}

\subsection*{Algoritmo de Interpolación de Transiciones}

Este algoritmo suaviza las transiciones entre segmentos de video interpolando los cuadros finales de un segmento y los cuadros iniciales del siguiente.

\begin{enumerate}
    \item \textbf{Ordenar Segmentos:} Similar al algoritmo de ordenación y concatenación:
    
    \[
    S_{\text{ordenado}} = \text{sort}(S, \text{by} \ p_i)
    \]
    
    \item \textbf{Calcular Interpolación entre Segmentos:} Para suavizar la transición entre dos segmentos \( S_i \) y \( S_{i+1} \), interpolamos los cuadros finales de \( S_i \) y los cuadros iniciales de \( S_{i+1} \).
    
    Sea \( F_i \) el último cuadro de \( S_i \) y \( I_{i+1} \) el primer cuadro de \( S_{i+1} \). La interpolación lineal entre estos cuadros para un conjunto de cuadros interpolados \( \{C_j\} \) se define como:
    
    \[
    C_j = \alpha_j F_i + (1 - \alpha_j) I_{i+1}
    \]
    
    donde \( \alpha_j \) es un coeficiente de interpolación que varía linealmente de 0 a 1 a lo largo del número de cuadros interpolados.
    
    \item \textbf{Concatenar Segmentos con Interpolación:} Concatenamos los segmentos junto con los cuadros interpolados \( \{C_j\} \):
    
    \[
    V = S_{\text{ordenado}}[1] + \{C_1, C_2, \ldots, C_k\} + S_{\text{ordenado}}[2] + \ldots + S_{\text{ordenado}}[n]
    \]
    
    donde \( k \) es el número de cuadros interpolados entre cada par de segmentos.
\end{enumerate}

\end{document}

